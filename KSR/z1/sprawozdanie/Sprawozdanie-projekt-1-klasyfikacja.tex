\documentclass{article}
\usepackage[T1]{fontenc}
\usepackage[utf8]{inputenc}
\usepackage[polish]{babel}
\usepackage{amsmath}
\usepackage{enumitem}
%{Informatyka stosowana 2020, I st., semestr VI}


\author{
	{Julia Ruszer, 247775} \\
	{Wiktor Żelechowski, 247833}\\ 
{Prowadzący: prof. dr hab. inż. Adam Niewiadomski}
}

\title{Komputerowe systemy rozpoznawania 2024/2025\\Projekt 1. Klasyfikacja dokumentów tekstowych}
\begin{document}
\maketitle



\section{Cel projektu}
Celem projektu jest stworzenie aplikacji do klasyfikacji zbioru dokumentów tekstowych \cite{reuters} do odpowiednich krajów metodą k-NN oraz analiza jej działania. Do wykonania tego zadania niezbędne będzie przeprowadzenie ekstrakcji cech. Następnie zbadany zostanie wpływ parametrów na jakość klasyfikacji.


\section{Klasyfikacja nadzorowana metodą $k$-NN.  Ekstrakcja cech, wektory cech}
\subsection{Klasyfikacja nadzorowana metodą $k$-NN}
Do klasyfikacji artykułów wykorzystany został algorytm $k$-NN, którego działanie jest opisane w \cite{tadeusiewicz90}. W naszym zadaniu obiekty ze zbioru \cite{reuters} będą przydzielane do jednej z następujących klas: USA, UK, Canada, Japan, West Germany, France; jedynie na podstawie wektorów cech (opisane w pkt \ref{sec:wektor}) \\
Algorytm działa w sposób deterministyczny, czyli jego wyniki sa zależne jedynie od parametrów.
Parametry wejściowe implementacji algorytmu to
\vspace{-2mm}
\begin{itemize} \itemsep1pt \parskip0pt \parsep0pt
  \item wartość $k$ - liczba szukanych najbliższych sąsiadów, wybierana w sposób arbitralny
  \item proporcje podziału zbioru wektorów cech na zbiór treningowy i testowy
  \item zbiór cech na podstawie których będzie dokonymana klasyfikacja
  \item metryka według której obliczana będzie odległość badanego obiektu od elementów ze zbioru treningowego
\end{itemize}\medskip
W wyniku działania klasyfikatora wszystkie artykuły ze zbioru treningowego zostaną przydzielone do jednej z wyżej wymienionych klas.
\subsection{Ekstrakcja cech} \label{sec:cechy}
Aby możliwa była klasyfikacja dokumentów tekstowych konieczne jest przeprowadzenie ekstrakcji cech. Większość z cech będzie znajdywana z wykorzystaniem wcześniej przygotowanych dwóch zbiorów istotnych słów - nazw wszystkich państw, które mogą pojawić się w tekście oraz nazw walut, które były wykorzystywane w państwach, dla których przeprowadzana jest klasyfikacja. Zbiory te zostały przygotowane w oparciu o dane udostępnione przez \cite{reuters}. Przy pomocy wcześniej przygotowanych zbiorów wyekstrahowane zostały następujące cechy:
\vspace{-2mm}
\begin{enumerate}[label=c\arabic*.]  \itemsep1pt \parskip0pt \parsep0pt
  \item Długość artykułu - liczba słów z których składa się tekst artykułu. Cecha ta przyjmuje wartości nieujemne, które należą do zbioru liczb całkowitych i pozwala na zwiększenie jakości klasyfikacji dokumentów, jeżeli dokumenty pochodzące z jednej klasy mają podobne długości i średnie długości tekstów różnią się w zależności od etykiety artykułu.
  \begin{equation} 
        {n} = |W| 
  \end{equation}

gdzie \(W\) - zbiór wszystkich słów w dokumencie
  \item Liczba nieunikalnych słów składających się tylko z wielkich liter - przyjmuje wartości liczbowe, które są całkowite i nieujemne. Cecha umożliwia zwiększenie jakości klasyfikacji dokumentów, jeżeli w dokumentach pewnej klasy jest więcej słów zawierających wyłącznie wielkie litery, co może wskazywać na częstsze używanie skrótów nazw własnych (np. UK, US, WHO).  

\begin{subequations}
\begin{align}
a &= \sum_{i=0}^{n} onlyCapital(w_i), \label{eq:main} \\
\text{gdzie } onlyCapital(w) &= 
\begin{cases} 
1, & \text{jeśli } w \text{ składa się tylko z wielkich liter} \\
0, & \text{w przeciwnym wypadku} \label{eq:cases}
\end{cases}
\end{align}
\end{subequations}
gdzie  \(i\) - indeks słowa w dokumencie, \( w_{i} \) - i-te słowo dokumentu, \(n\) - liczba wszystkich słów w dokumencie, \(w\) - pojedyncze słowo dokumentu

  \item Liczba nieunikalnych słów rozpoczynających się wielką literą - przyjmuje wartości liczbowe, które są całkowite i nieujemne.
  \begin{subequations} 
\begin{align}
b &= \sum_{i=0}^{n} firstCapital(w_i), \label{eq:main} \\
\text{gdzie } firstCapital(w) &= 
\begin{cases} 
1, & \text{jeśli } w \text{ rozpoczyna się wielką literą} \\
0, & \text{w przeciwnym wypadku} \label{eq:cases}
\end{cases}
\end{align}
\end{subequations}

gdzie  \(i\) - indeks słowa w dokumencie, \( w_{i} \) - i-te słowo dokumentu, \(n\) - liczba wszystkich słów w dokumencie, \(w\) - pojedyncze słowo dokumentu
  \item Pierwsze słowo, które jest nazwą waluty - wartością ekstrakcji jest pojedyncze słowo, które jest nazwą waluty. Cecha pozwala zwiększyć jakość klasyfikacji poprzez znalezienie waluty charakterystycznej dla danej klasy.
  \begin{equation}
      cur_{first} = w_i, \; \text{gdzie} \; i = \min \{ j: w_j \in W \land w_j \in V \}
  \end{equation}
gdzie \( cur_{first} \) - pierwsze słowo, które jest nazwą waluty, \( w_{i} \) - i-te słowo artykułu, \(i\) - indeks słowa w dukumencie, \(W\) - zbiór wszystkich słów artykułu, \(V\) - zbiór słów, które są nazwą waluty
  \item Pierwsze słowo, które jest nazwą państwa - wartością ekstrakcji jest pojedyncze słowo, które jest nazwą państwa. Cecha może zwiększyć jakość klasyfikacji, ponieważ pierwsze wymienione państwo w artykule może mieć największy związek z klasą obiektu. 
  \begin{equation}
      cntr_{first} = w_i, \; \text{gdzie} \; i = \min \{ j: w_j \in W \land w_j \in P \}
  \end{equation}

gdzie \( cntr_{first} \) - pierwsze słowo, które jest nazwą państwa, \( w_{i} \) - i-te słowo artykułu, \(i\) - indeks słowa w dukumencie, \(W\) - zbiór wszystkich słów artykułu, \(P\) - zbiór słów, które są nazwą państwa
  \item Najczęściej występujące słowo, które jest nazwą państwa - wartością ekstrakcji jest pojedyncze słowo, które jest najczęściej pojawiającą się nazwą państwa w artykule.Cecha może zwiększyć jakość klasyfikacji, ponieważ najczęściej wymieniane państwo w artykule może mieć duży związek z klasą obiektu. 
  \begin{equation}
  c = \{w \in W \cap P: count(w) = max \{count(z): z \in W \land z \in P\}\}
  \end{equation} 
gdzie \( count(w) \) - funkcja zliczająca liczbę wystąpień słowa w dokumencie, \(W\) - zbiór wszystkich słów artykułu, \(P\) - zbiór słów, które są nazwą państwa

  \item Łączna liczba wystąpień słów, które są unikalnymi nazwami państw - przyjmuje wartości liczbowe, które są całkowite i nieujemne. Ta cecha może zwiększyć jakość klasyfikacji przy założeniu, że w artykułach różnych klas pojawia się średnio inna liczba wymienianych różnych państw.
  \begin{subequations} 
\begin{align}
auc &= \sum_{i=0}^{n_u} checkCountry(uw_i), \\
\text{gdzie } checkCountry(uw) &= 
\begin{cases} 
1, & \text{jeśli } uw \in P \\
0, & \text{jeśli } uw \notin P
\end{cases}
\end{align}
\end{subequations}
gdzie  \(i\) - indeks słowa w dokumencie, \(n_u\) - liczba unikalnych słów w dokumencie, \(P\) - zbiór słów, które są nazwą państwa, \(uw_{i}\) - i-te słowo ze zbioru unikalnych słów dokumentu, \(uw\) - pojedyncze słowo ze zbioru unikalnych słów dokumentu

  \item Łączna liczba wystąpień wszystkich słów, które są nazwami państw - przyjmuje wartości liczbowe, które są całkowite i nieujemne. Cecha ta może zwiększyć jakość klasyfikacji, jeżeli jedne klasy artykułów posiadają więcej wystąpień nazw państw niż inne (liczymy powtórzenia słów).
   \begin{equation}
       ac = \{| w: w \in W \land w \in P |\}
   \end{equation}

gdzie \(W\) - zbiór wszystkich słów artykułu, \(P\) - zbiór słów, które są nazwą państwa
  \item Łączna liczba wystąpień słów, które są nieunikalnymi nazwami państw albo walut - przyjmuje wartości liczbowe, które są całkowite i nieujemne. Cecha ta może zwiększyć jakość klasyfikacji, jeżeli jedne klasy artykułów posiadają więcej wystąpień poszukiwanych słów niż inne (liczymy powtórzenia słów).
   \begin{equation}
       k = \{| w: w \in W \land (w \in P \lor w \in V)|\}
   \end{equation}

gdzie \(W\) - zbiór wszystkich słów artykułu, \(P\) - zbiór słów, które są nazwą państwa, \(V\) - zbiór słów, które są nazwą waluty
  \item Względna liczba wystąpień nieunikalnych nazw państw - przyjmuje wartości liczbowe z zakresu [0; 1]. Oznacza stosunek liczby wystąpień nieunikalnych nazw państ do liczby wszystkich słów w artykule. Cecha ta może zwiększyć jakość klasyfikacji, jeżeli w niektórych klasach artykułów częstotliwość wymieniania nazw państw jest większa niż w innych.
\begin{equation}
     rc = \frac{\left| \{ w : w \in W \land w \in P \} \right|}{|W|} 
\end{equation}

gdzie \(W\) - zbiór wszystkich słów artykułu, \(P\) - zbiór słów, które są nazwą państwa
\end{enumerate}\medskip
\subsection{Wektory cech} \label{sec:wektor}
Dla każdego artykułu po wyekstrahowaniu cech otrzymaliśmy wektor cech, który przyjmuje postać \textit{v = [c1, c2, c3, c4, c5, c6, c7, c8, c9, c10]}, gdzie \textit{c1, c2, ..., c10} to wartości uzyskane podczas ekstrakcji cech w sposób opisany w \ref{sec:cechy}. Każdy tak otrzymany wektor cech reprezentuje pojedynczy element ze zbioru wszystkich artykułów i na jego podstawie dokosywana będzie klasyfikacja dokumentów.



\section{Miary jakości klasyfikacji} 
Do oceniania jakości klasyfikacji artykułów skorzystaliśmy z czterech miar jakości klasyfikacji - accuracy (\cite{tabpom}), precision (\cite{tabpom}), recall (\cite{tabpom}) oraz F1 (\cite{miary}).
\subsection{Accuracy}
Accuracy określa dokładność całego procesu klasyfikacji i jest obliczana na podstawie wyników klasyfikacji dla wszystkich klas łącznie. 
\\Miara określana jest w procentach i może przyjąć wartości z zakresu od 0\% (jeśli wszystkie artykuły zostały niepoprawnie przypisane do klas) do 100\% (jeśli wszystkie artykuły zostały poprawnie przypisane do klas).
\begin{equation}
     accuracy = \frac{\left|TA \right|}{\left| A\right|} \cdot 100\%
\end{equation}
gdzie \(TA\) - zbiór wszystkich poprawnie sklasyfikowanych artykułów, \(A\) - zbiór wszystkich sklasyfikowanych artykułów
\subsection{Precision} \label{sec:precision}
Precision określa precyzję procesu klasyfikacji w obrębie jednej klasy. Ta miara pozwala określić prawdopodobieństwo na to, że artykuł przypisany do danej klasa, faktycznie z niej pochodzi. 
\\Może przyjąć wartości z przedziału [0; 1], gdzie wartość 0 oznacza, że wszystkie artykuły skalsyfikowane jako pochodzące z danej klasy w rzeczywistości należą do innej, zaś wartość precyzji 1 oznacza, że wszystkie artykuły sklasyfikowane jako pochodzące z danej klasy faktycznie z niej pochodzą.
\begin{equation} \label{eq:precision}
     precision_K = \frac{\left|TK \right|}{\left|TK\right| + \left|FK\right|}
\end{equation}
gdzie \(TK\) - zbiór poprawnie sklasyfikowanych artykułów z klasy K, \(FK\) - zbiór artykułów, które niepoprawnie zostały sklasyfikowane jako klasa K
\subsection{recall} \label{sec:recall}
Recall określa czułość procesu klasyfikacji dla danej klasy. Ta miara pozwala określić prawdopodobieństwo poprawnego sklasyfikowania artykułu pochodzącego z tej klasy. 
\\Może przyjąć wartości z przedziału [0; 1], gdzie wartość 0 oznacza, że żaden artykuł z danej klasy nie został poprawnie sklasyfikowany, zaś wartość 1 oznacza, że każdy artykuł, który pochodzi z tej klasy został poprawnie sklasyfikowany.
\begin{equation} \label{eq:recall}
     recall_K = \frac{\left|TK \right|}{\left|K\right|}
\end{equation}
gdzie \(TK\) - zbiór poprawnie sklasyfikowanych artykułów z klasy K, \(K\) - zbiór wszystkich artykułów, które powinny zostać sklasyfikowane jako klasa K
\subsection{F1}
F1 określa skuteczność klasyfikacji w obrębie jednej klasy, uwzględniając dwie inne miary - precyzję (pkt \ref{sec:precision}) oraz czułość (pkt \ref{sec:recall}). Wskaźnik ten ma największe znaczenie w przypadku dużych różnic w wynikach dwóch wcześniej wspomnianych miar, ponieważ uśrednia te wartości ułatwiając porównywanie wyników klasyfikacji.
\\Może przyjąc wartości z przedziału [0; 1], gdzie wartość 0 oznacza najniższą możliwą skuteczność (jeśli co najmniej 1 z uwzględnianych miar przyjmuje wartość 0), zaś 1 - najwyższą (jeśli obie uwzględniane miary przyjmują wartość 1). 
\begin{equation}
     F1_K = 2 \cdot \frac{precision_K \cdot recall_K}{precision_K + recall_K}
\end{equation}
gdzie \(\textit{precision}_K\) - wzór~\ref{eq:precision}, \(\textit{recall}_K\) - wzór~\ref{eq:recall}

\section{Metryki i miary podobieństwa tekstów w klasyfikacji}
Wzory, znaczenia i opisy symboli zastosowanych metryk z
przykładami. Wzory, opisy i znaczenia miar
podobieństwa tekstów zastosowanych w obliczaniu metryk dla wektorów cech z
przykładami dla każdej miary \cite{niewiadomski08}.  Oznaczenia jednolite w obrębie całego sprawozdania.  {\bf Podaj metryki i miary
podobieństwa nie z literatury (te wystarczy zacytować linkiem), ale konkretne ich
postaci stosowane w zadaniu. Jakie zakresy wartości przyjmują te miary i
metryki, co oznaczają ich wartości? Podaj przykładowe wartości dla przykładowych wektorów cech}. \\ 
\noindent {\bf Sekcja uzupełniona jako efekt zadania Tydzień 04 wg Harmonogramu Zajęć
na WIKAMP KSR.}

\section{Wyniki klasyfikacji dla różnych parametrów wejściowych}
Wstępne wyniki miary Accuracy dla próbnych klasyfikacji na ograniczonym zbiorze tekstów (podać parametry i kryteria
wyboru wg punktów 3.-8. z opisu Projektu 1.). 
\noindent {\bf Sekcja uzupełniona jako efekt zadania Tydzień 05 wg Harmonogramu Zajęć
na WIKAMP KSR.}


\section{Dyskusja, wnioski, sprawozdanie końcowe}

Wyniki kolejnych eksperymentów wg punktów 2.-8. opisu projektu 1.  Każdorazowo
podane parametry, dla których przeprowadzana eksperyment. 
Wykresy (np. słupowe) i tabele wyników
obowiązkowe, dokładnie opisane w ,,captions'' (tytułach), konieczny opis osi i
jednostek wykresów oraz kolumn i wierszy tabel.\\ 

{**Ewentualne wyniki realizacji punktu 9. opisu Projektu 1., czyli ,,na ocenę 5.0'' i ich porównanie do wyników z
części obowiązkowej**.Dokładne interpretacje uzyskanych wyników w zależności od parametrów klasyfikacji
opisanych w punktach 3.-8 opisu Projektu 1. 
Omówić i wyjaśnić napotkane problemy (jeśli były). Każdy wniosek/problem powinien mieć poparcie
w przeprowadzonych eksperymentach (odwołania do konkretnych wyników: wykresów,
tabel). \\
\underline{Dla końcowej oceny jest to najważniejsza sekcja} sprawozdania, gdyż prezentuje poziom
zrozumienia rozwiązywanego problemu.\\

** Możliwości kontynuacji prac w obszarze systemów rozpoznawania, zwłaszcza w kontekście pracy inżynierskiej,
magisterskiej, naukowej, itp. **\\

\noindent {\bf Sekcja uzupełniona jako efekt zadań Tydzień 05 i Tydzień 06 wg Harmonogramu Zajęć
na WIKAMP KSR.}


\section{Braki w realizacji projektu 1.}
Wymienić wg opisu Projektu 1. wszystkie niezrealizowane obowiązkowe elementy projektu, ewentualnie
podać merytoryczne (ale nie czasowe) przyczyny tych braków. 


\begin{thebibliography}{0}
\bibitem{reuters} Reuters-21578 Text Categorization Collection [online]. [dostęp 14.03.2025]. Dostępny w internecie: http://archive.ics.uci.edu/dataset/137/reuters+21578+text+categorization+collection
\bibitem{tadeusiewicz90} R. Tadeusiewicz: Rozpoznawanie obrazów, PWN, Warszawa, 1991  
\bibitem{horzykagh} A. Horzyk, Metody Inteligencji Obliczeniowej Metoda K Najbliższych Sąsiadów (KNN) [online]. [dostęp 14.03.2025]. Dostępny w internecie: https://home.agh.edu.pl/~horzyk/lectures/miw/KNN.pdf
\bibitem{tabpom} Tablica pomyłek [online]. [dostęp 19.03.2025]. Dostępny w internecie: https://pl.wikipedia.org/wiki/Tablica\_pomyłek
\bibitem{miary} Wyniki modeli uczenia maszynowego [online]. [dostęp 21.03.2025]. Dostępny w internecie: https://learn.microsoft.com/pl-pl/dynamics365/finance/finance-insights/confusion-matrix
\end{thebibliography}

\end{document}
